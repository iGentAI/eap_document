\subsection*{User Menu}

\begin{center}
\includegraphics[width=0.4\textwidth]{user_menu.png}
\end{center}

The user menu provides access to essential settings and configuration options for your Maestro session.

\subsubsection*{Light/Dark Mode}

Toggle between light and dark themes to match your preferred working environment. The interface automatically adjusts colors and contrast to ensure comfortable viewing in any lighting condition.

\subsubsection*{Link GitHub}

Connect your GitHub account to enable Maestro's source control management capabilities. This authentication allows the agent to:
\begin{itemize}
    \item Work with your private repositories
    \item Create new repositories under your account or organizations
    \item Generate pull requests on your behalf
    \item Clone and manage both public and private repositories
\end{itemize}

Once linked, Maestro can seamlessly integrate with your GitHub workflow, making version control operations as simple as natural language commands.

\subsubsection*{Manage Credentials}

The credential management system allows you to securely store and use:
\begin{itemize}
    \item API keys for third-party services
    \item Environment variables for your applications
    \item Authentication tokens and secrets
    \item Complete \texttt{.env} files for project configurations
\end{itemize}

\textbf{Important}: To use stored credentials in your session, you must explicitly activate them by either:
\begin{itemize}
    \item Using the \texttt{/credentials} command
    \item Asking the agent to activate specific credentials when needed
\end{itemize}

This explicit activation ensures your sensitive information is only exposed when necessary and under your direct control.